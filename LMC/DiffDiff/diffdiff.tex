\documentclass[11pt]{article}
\usepackage{indentfirst}
\usepackage{graphicx}

\usepackage{algorithmicx}
\usepackage[ruled]{algorithm}
\usepackage{algpseudocode}
\alglanguage{pseudocode}


\topmargin 0.0 in
\rightmargin 0.0 in
\oddsidemargin 0.0 in
\evensidemargin 0.0 in
\textwidth 6.5 in
\textheight 8.5 in

\begin{document}

\begin{abstract}
A multigrid algorithm is discribed below as implemented in LMC to solve the discrete equations
associated with multicomponent diffusion.
\end{abstract}

\newcommand{\gp}{{\vec \imath}}
\newcommand{\told}{{t^{n}}}
\newcommand{\tnew}{{t^{n+1}}}
\newcommand{\nua}{{\nu_1}}
\newcommand{\nub}{{\nu_2}}

\section{Model}
In the zero Mach number limit with constant ambient pressure,
the equations describing momentum transport
and conservation of species and enthalpy are given by
\begin{equation}
\frac{\partial (\rho U)}{\partial t} + \nabla \cdot \rho UU \nabla \cdot \tau = -\nabla \pi 
- \rho g \enskip , \label{eq:mom}
\end{equation}
\begin{equation}
\frac{\partial (\rho Y_m)}{\partial t} + \nabla \cdot \rho U Y_m + \nabla \cdot {\cal F}_m
= \rho \dot{\omega}_m \enskip , \label{eq:species}
\end{equation}
\begin{equation}
\frac{ \partial (\rho h)}{ \partial t} + \nabla \cdot \rho U h + \nabla \cdot {\cal Q} = 0 \enskip ,
\label{eq:enthalpy}
\end{equation}
where $\rho$ is the density, $U$ is the velocity, $\vec{g}$ is the
gravitational acceleration vector, $Y_m$ is the mass fraction of
species $m$, $h$ is the enthalpy of the fluid, $\tau$ is the stress
tensor, and ${\cal {F}}_m,$ $\dot{\omega}_m$ represent the diffusion
flux and mass production rate and species $m$ respectively, and $Q$ is
heat flux.  Neither species diffusion nor reactions redistribute total
mass, i.e.\ $\sum {\cal F}_{m} = \sum {\dot \omega}_{m} = 0$, and since
$\sum_{m} Y_m = 1$, continuity is implied as the sum of equation
(\ref{eq:species}):
\begin{equation}
\frac{\partial \rho}{\partial t} + \nabla \cdot  (\rho U) = 0 \enskip .
\label{eq:cont}
\end{equation}
The production rates, $\rho \dot{\omega}_m$, due to chemical reactions
are specified via a collection of fundamental reactions in a
CHEMKIN-III compatible database.%\cite{KeeRupley1996}

The molecular transport fluxes in equations (\ref{eq:species}) and
(\ref{eq:enthalpy}) are given by kinetic theory in terms of
macroscopic variable gradients and transport coefficients:
\begin{eqnarray}
{\cal F}_{m} &=& \sum \rho Y_{m} {\cal D}_{mn} {\bf {\it d}}_{n} - \rho Y_{m}\theta_{m} \nabla \log T, \nonumber \\
{\cal Q} &=& \sum h_{m} {\cal F} - \lambda \nabla T - p \sum \theta_{m} {\bf {\it d}}_{m}
\label{eq:gensd}
\end{eqnarray}
Here ${\cal D}_{mn}$ are the binary mass diffusion coefficients and
$\theta_{m}$ are the thermal diffusion coefficients for species $m$.
In this model, ${\bf {\it d}}_{m}$ are the species diffusion driving
forces, including mole fraction gradients, pressure gradients and
differences between species-specific ``body'' forces (such as electric
fields or inertial frame effects).  In the absence of these latter
effects and in the zero Mach number limit, ${\bf {\it d}}_{m} = \nabla
X_{m}$.  The second term in the expression for ${\cal F}_{m}$ is the
Soret effect, and represents species transport due to temperature
gradients.  The last term in the definition of $\cal Q$ is the Dufour
effect, which represents heat diffusion due to concentration
gradients.% See, for example \cite {Ern1994,Ern1996}.

For many practical combustion systems, the Dufour and Soret
contributions are negligible and nitrogen typically dominates the gas
mixtures.  In such cases, the multicomponent transport can be
well-approximated by a simpler ``mixture-averaged'' Fickian model.
However, when the combustion fuel contains significant amounts of
$H_2$, Soret effects can be significant
%(see, e.g.\ \cite{Grcar2009,Grcar2009a}),
and the general model described by
equation (\ref{eq:gensd}) is more appropriate.  We describe a
semi-implicit discretization of the full multicomponent diffusion
system based on the full approximation scheme (FAS) 
%\cite{Brandt1977}.
FAS requires a ``relaxation''
procedure to iteratively smooth the solution at the various multigrid
levels.  We discuss a relatively simple, but effective smoother
for the full transport model based on the diagonal mixture model.

\section{FAS multigrid}
The (nonlinear) equation
\begin{equation}
   {\cal D}^{\ell} = {\cal R}^{\ell} - {\cal L}({\phi}^{\ell}) = 0 
  \label{eqn:NLupdate}
\end{equation}
defines an update rule for a class of time-implicit discretizations of
the multicomponent diffusive transport model.  The state of the system
at the ``new'' time, $\tnew$, on multigrid level $\ell$ is given by
${\phi}^{\ell}$, and includes the temperature, $T$, specific enthalpy,
$h$, mass density, $\rho$, and gas composition, expressed in terms of
species mass fraction, $Y_{j}$, for species $j\in{1,N_{s}}$.  The
levels, $\ell>0$, are generated by the iterative process discussed
below, and are used solely as an aid to solving for the true unknowns,
${\phi}^{0}$.  This discrete update is nonlinear and cannot be
represented as a simple matrix-vector multiply.  A related system,
${\cal D}^{\prime,\ell} = {\cal R}^{\prime,\ell}({\phi}^{\prime,\ell})
- {\cal L}({\phi}^{\prime,\ell}) = 0$, defines an update that
approximates the full multicomponent equations, but is more ammenable
to building the necessary components of an iterative solution
technique.  In particular, the approximation is constructed based on a
simpler mixture-averaged transport model.  Under a suitable
linearization (eg., freezing the transport coefficients and the
species enthalpies that are transported with the mass diffusion
fluxes), the approximate operator can be expressed as a matrix
multiply, ${\cal M} \cdot {\phi}^{\ell}$, such that there are no
elements in ${\cal M}$ which couple the various components of $\phi$,
and a simple smoother for the approximate system
(a so-called {\it Jacobi iteration}) can be constructed.  For
component $n$ of ${\phi}^{\ell}$ at grid point $\gp$ \ :
\begin{equation}
       {\phi}^{\prime,\ell}(\gp,n) \leftarrow {\phi}^{\prime,\ell}(\gp,n) + {\cal C}^{\ell}(\gp,n), \hspace{1cm}
                {\cal C}^{\ell}(\gp,n) = f \cdot {\cal D}^{\prime,\ell}(\gp,n) / {\alpha}^{\prime,\ell}(\gp,n).
  \label{eqn:linUpdate}
\end{equation}
Here $f$ is an (arbitrary) ``under-relaxation'' coefficient, and
${\alpha}^{\prime,\ell}(\gp,n)$ is the diagonal matrix entry of ${\cal
  M}$ at for component $n$ at grid point $\gp$.  For simplicity, we'll
drop the $(\gp,n)$ notation for much of the following and assume the
operation is applied over the entire grid at level $\ell$.  The main
idea to explore in this work is whether the relaxation coefficent in
Equation~(\ref{eqn:linUpdate}) can be used to form a smoother as the
basis of an FAS multigrid approach to solve
Equation~(\ref{eqn:NLupdate}).

The precise diffusion transport model and the discrete equation for
its update is discussed elsewhere, but can be written formally as:
\[
  {\cal L}({\phi}^{\ell}) \approx {\rho}^{\ell} {\phi}^{\ell} - {\cal V}({\phi}^{\ell})
\]
where ${\cal V}({\phi}^{\ell})$ is the divergence of diffusion fluxes,
scaled by $\theta \, dt$, $dt = \tnew - \told$ is the discrete time
interval of the update, and $\theta$ is an implicitness parameter
($\theta = 0.5$ for Crank-Nicolson and $\theta=1$ for backward Euler).
In this case, ${\cal V}$ represents a conservative difference for
updating $\rho Y_{j}$ and $\rho h$.

There are several observations relevant to the software implementation to be discussed.
\begin{itemize}
\item The operator, ${\cal V}({\phi}^{\ell})$, consists in part of
  evaluating edge-based fluxes of mass and energy as a functions of
  the solution, ${\phi}^{\ell}$.  This involves computation of
  edge-based diffusivities that are complex (and expensive) functions
  of the local species concentrations and temperature fields.  It is
  convenient to evaluate the relaxation coefficients,
  ${\alpha}^{\ell}$, at the same time ${\cal V}({\phi}^{\ell})$ is
  computed.  It is also useful to separate the evaluation of these
  coefficients and the operator.  We can then explore the frequency of
  updates necessary for these quantities as an optimzation.
\item Since the approximate operator matrix, ${\cal M}$, is also a
  function of species and temperature, the smoothing operator is most
  conveniently expressed in terms of updates to $T$ (rather than
  $\rho h$), although either can be constructed through appropriate
  linearizations.  It is not yet clear which linearization provides
  the best behaved relaxation scheme.  It is also not yet clear when,
  during the iterations, it makes sense to synchronize the energy
  variables.  For example, when the conserved variables are coarsened
  as part of the FAS scheme below, there may not be a reasonable value
  of $T$ consistent with the resulting state.  Likewise, one may
  devise methods to coarsen $T$ and $Y$, but then find that these lead
  to an unusable value of $h$.  The option to work with either one is
  provided at the outset, as is the option to synchronize the two.
  Some testing is called for.
\item The relaxation is modified to account for changes to the stencil
  of the approximate operator near physical boundaries.  In LMC, grow
  cells are used to separate and compartmentalize the specifics of the
  boundary condition implementation, but care is taken to retain the
  effective influence of the central stencil component in computing
  $\alpha$ at each location.
\end{itemize}

\newpage
The current version of the FAS cycle can be written in pseudo code as follows:

\begin{algorithm}[H]
\caption{Algorithm to compute ${S}^{n+1}$ from ${S}^{n}$.}
\begin{algorithmic}
\Procedure{MCDDUpdate}{${S}^{n},{\cal A}^{n},\theta,dt,\nu_{0},\nu_{1},\nu_{2},k$} 
   \Comment{${\cal A}^{n}$ is a forcing term, constant over $dt$}
   \State ${\phi}^{0}_{0} = ({S}^{n} + dt \cdot {\cal A}^{n})/ {\rho}^{0}$
   \State ${R}^{0} = {S}^{n} + (1-\theta)\cdot dt \cdot {\cal V}({\phi}^{0}_{0})
                           + dt \cdot {\cal A}^{n}$
   \For{$j = 1,\, \nu_{0}$}
      \State $({\phi}^{0}_{j},{\cal D}^{0}_{j}) = \mbox{\sc FASCycle}({\phi}^{0}_{j-1},{\cal R}^{0},\nua,\nub,k)$
 \Comment{small ${\cal D}^{0}_{j} \rightarrow$ {\bf solved} ($j=\nu_{0}$, exit loop)}
   \EndFor
   \State Synchronize $h,Y,T$
   \State ${S}^{n+1} = {\phi}^{0}_{\nu0} {\rho}^{0}$
   \State return ${S}^{n+1}$
\EndProcedure
\end{algorithmic}
\end{algorithm}


\begin{algorithm}[H]
\caption{FAS algorithm to solve ${\cal L}({\phi}^{\ell}) = {\cal R}^{\ell}$ on
       level $\ell$ (begins with ${\phi}^{\ell}_{0}$ and returns ${\phi}^{*,\ell}_{\nub}$).
}
\begin{algorithmic}
\Procedure{FASCycle}{${\phi}^{\ell}_{0},{\cal R}^{\ell},\nua,\nub,k$}
   \For{$j = 1,\, \nu_1$}
      \State $({\cal L} ({\phi}^{\ell}_{j-1}), \alpha^{\ell}_{j-1})
             \gets {\cal V}(\phi^{\ell}_{j-1})$
      \State ${\cal D}^{\ell}_{j-1} = {\cal R}^{\ell} - {\cal L}({\phi}^{\ell}_{j-1})$
          \Comment{small ${\cal D}^{\ell}_{j-1} \rightarrow$ {\bf solved} (return $\phi^{\ell}_{j-1}$)}
      \State ${\cal C}^{\ell}_{j-1} = f \cdot {\cal D}^{\ell} / {\alpha}^{\ell}_{j-1}$ 
          \Comment{small ${\cal C}^{\ell}_{j-1} \rightarrow$ {\bf stalled} ($j=\nua$, exit loop)}
      \State ${\phi}^{\ell}_{j} = {\phi}^{\ell}_{j-1} - {\cal C}^{\ell}_{j-1}$
   \EndFor
   \State ${\cal D}^{\ell}_{\nua} = {\cal R}^{\ell} - {\cal L} ({\phi}^{\ell}_{j})$
          \Comment{small ${\cal D}^{\ell}_{\nua} \rightarrow$ {\bf solved} (return $\phi^{\ell}_{\nua}$)}
   \State
   \If{$k > \ell + 1$}
      \State ${\phi}^{\ell+1}_{0} = \mbox{\sc AvgDown}({\rho}^{\ell} {\phi}^{\ell}_{\nua}) / {\rho}^{\ell+1}$
      \State ${\cal R}^{\ell+1} = \mbox{\sc AvgDown}({\cal D}^{\ell}_{\nua}) + {\cal L} ({\phi}^{\ell+1}_{0})$
      \State ${\phi}^{*,\ell+1}_{save} = {\phi}^{\ell+1}_{0}$
      \State ${\phi}^{*,\ell+1}_{\nub} = \mbox{\sc FASCycle}({\phi}^{\ell+1}_{0},{\cal R}^{\ell+1},\nua,\nub,k-1)$
      \State ${e}^{\ell+1} = {\phi}^{*,\ell+1}_{\nub} - {\phi}^{*,\ell+1}_{save}$
      \State ${\phi}^{*,\ell}_{0} = {\phi}^{\ell}_{\nua} + \mbox{\sc Interp}({\rho}^{\ell+1} {e}^{\ell+1}) / {\rho}^{\ell})$
   \EndIf
   \State
   \For{$j = 1,\, \nu_2$}
      \State $({\cal L} ({\phi}^{*,\ell}_{j-1}), \alpha^{*,\ell}_{j-1})
             \gets {\cal V}(\phi^{*,\ell}_{j-1})$
      \State ${\cal D}^{*,\ell}_{j-1} = {\cal R}^{\ell} - {\cal L}({\phi}^{*,\ell}_{j-1})$
          \Comment{small ${\cal D}^{*,\ell}_{j-1} \rightarrow$ {\bf solved} (return $\phi^{*,\ell}_{j-1}$)}
      \State ${\cal C}^{*,\ell}_{j-1} = f \cdot {\cal D}^{*,\ell} / {\alpha}^{*,\ell}_{j-1}$ 
          \Comment{small ${\cal C}^{*,\ell}_{j-1} \rightarrow$ {\bf stalled} ($j=\nub$, exit loop)}
      \State ${\phi}^{*,\ell}_{j} = {\phi}^{*,\ell}_{j-1} - {\cal C}^{*,\ell}_{j-1}$
   \EndFor
   \State return $\phi^{*,\ell}_{\nub}$
           \Comment{Returns improved ${\phi}^{\ell}$}
\EndProcedure
\end{algorithmic}
\end{algorithm}

Additional notes:
\begin{itemize}
\item A {\bf stall} in the solution process is detected when
  corrections, ${\cal C}(\gp,n)$, are small with respect to a typical
  value, $\mbox{\it typ}_n$ (ie, ${\cal C}(\gp,n) <= \epsilon_c
  \mbox{\it typ}_n$, where $\epsilon_c$ is a small number).  The array
  of typical values is set at the start of a simulation by scanning
  each component over the entire domain for the largest absolute
  value.  These values are written in the checkpoint folder and are
  read in on restart in order that continuation runs remain
  consistent.
\item A small norm of the residual (or ``defect'' in FAS terminology),
  ${\cal D}(\gp,n)$ is detected when its value is smaller than that of
  the initial residual for that problem by a factor $\epsilon_{r0}$.
  In the present configuration, {\sc FASCycle} is called multiple
  times for the nonlinear system at $\ell=0$.  For each of these
  calls, a new problem is generated on each coarsened multigrid level,
  $\ell < k$.  In this version of the algorithm, the depth of the
  hierarchy may decrease during the iterations as the residual at one
  level becomes sufficiently small that the solver is exited prior to
  generating the coarser problem.  At the moment, this seems less than
  ideal since the solver often reduces to simple single-level
  relaxation before the $\ell = 0$ problem is completely solved.
  There is probably a level-adaptive way to define the residual so
  that the full hierarchy continues to be useful.
\item LMC is configured at the moment to allow an array of $\nua$ and
  $\nub$ values.  When $\ell=k$ (ie, the ``bottom solve''), $\nua =
  \nu_b$ and $\nub = 0$.
\item Additionally, one may optionally bail from the $\nua$ or $\nub$
  loops if the residual has been reduced by some factor,
  $\epsilon_{r1}$. $\epsilon_{r2}$, respectively.
\item A small change should allow this to be called as an FMG cycle,
  NLMG, or a W-cycle, if necessary.
\end{itemize}

\end{document}
