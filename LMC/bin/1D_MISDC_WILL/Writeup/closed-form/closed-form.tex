\documentclass[12pt]{article}
\usepackage{graphicx}
\usepackage{amsmath}
\usepackage{amssymb}
\usepackage{pifont}
\usepackage[margin=0.65in]{geometry}
\usepackage{enumerate}
\usepackage{caption}
\usepackage{float}
\usepackage{graphicx}
\usepackage{subfig}\usepackage{graphicx}
\usepackage{subfig}
\usepackage{bm}

\def\arraystretch{1.5}
\captionsetup{aboveskip=8pt}
\newcommand{\cmark}{\ding{51}}%
\newcommand{\xmark}{\ding{55}}%
\setlength{\parskip}{1em}

\begin{document}
\noindent

Recalling from Michael's notes, equations (24-25), using Forward Euler for the 
advection correction, we have $c^{(k)}_A = 0$. 
Additionally, we notice that if the residual 
$\bm{\tilde R}(\phi^{(k)},t_{n+1}) = 0,$ then we have that the term 
$c^{(k)}_{AD} = 0$ satisfies the Backward Euler equation
\begin{equation}
   c_{AD}^{(k)} = \Delta t\left(d(u^{(k),n+1} + c_{AD}^{(k)}) 
                              - d(u^{(k),n+1})\right)
                  + \bm{\tilde R}(\phi^{(k)},t_{n+1}).
\end{equation}
Therefore, since the correction is given by the ODE
\begin{align}
   (c^{(k)})'(t) &= ac_A(t) + dc_{AD}(t) + rc^{(k)}(t) + \bm{\tilde R}'\\
                 &= rc^{(k)}
\end{align}
with zero initial conditions, we have that $c^{(k)}(t) = 0$. Therefore, 
$\phi^{(k+1)} = \phi^{(k)}$.

Additionally, the update ODE for the new solution is given by
\begin{align}
   (c^{(k)} + \phi^{(k)})'(t) &= ac^{(k)}_A(t) + dc^{(k)}_{AD}(t) 
                                 + r(c^{(k)} + \phi^{(k)})(t)
                                 + \ell_k(t)\\
                              &= r(c^{(k)} + \phi^{(k)})(t) + \ell_k(t)
\end{align}
We write $\phi^{(k+1)} = c^{(k)} + \phi^{(k)}$. We will additionally assume 
$t_n = 0$. This is not an essential assumption, it entirely for the sake of 
simplicity. Since $\ell_k(t)$ is a linear approximation, we can write
\begin{equation}
   \ell_k(t) = mt + b.
\end{equation}
Our update ODE is therefore
\begin{equation}
   \phi^{(k+1)}_t(t) = r\phi^{(k+1)}(t) + mt + b, \qquad \phi^{(k+1)}(0) = \phi_n
\end{equation}
the solution to which is given by
\begin{equation}
   \label{ode-soln}
   \phi^{(k+1)}(t) = \frac{1}{r^2}\left(
      e^{rt}r^2 \phi_n + be^{rt}r + e^{rt}m -mrt - m
   \right)
\end{equation}

Recall that $\ell_k$ is given by
\begin{equation}
   \ell_k(t) = (a+d)\left(\phi_n\frac{t_{n+1}-t}{\Delta t} 
                        + \frac{t-t_n}{\Delta t}\phi^{(k)}_{n+1} \right).
\end{equation}
Since $\phi^{(k+1)} = \phi^{(k)}$, we must have
\begin{equation}
   \ell_k(t) = (a+d)\left(\phi_n\frac{t_{n+1}-t}{\Delta t} 
                        + \frac{t}{\Delta t}\phi^{(k+1)}_{n+1} \right).
\end{equation}
In other words,
\begin{equation}
   b= (a+d)\phi_n,\qquad
   m= \frac{(a+d)\left(\phi^{(k+1)}_{n+1} - \phi_n\right)}{\Delta t}.
\end{equation}
We can then write
\begin{equation}
   \label{m-eqn}
   \phi^{(k+1)}_{n+1} = \frac{m \Delta t}{a+d} + \phi_n.
\end{equation}
Solving simultaneously equations (\ref{ode-soln}) and (\ref{m-eqn}), we see 
that
\begin{equation}
   m = \frac{\phi_n}{\frac{\Delta t}{(a-d)(e^{r\Delta t} -1)} - \frac{1}{r(a+d+r)}}.
\end{equation}
Using this expression to substitute for $m$ and $b$ in equation (\ref{ode-soln}), 
we find that
\begin{equation}
   \phi^{(k+1)}(t) = \frac{\phi_n (a+d) \left(e^{\Delta t r} 
      (t (a+d+r)+1)-(t-\Delta t) (a+d+r)-1\right)-\Delta t 
      \phi_n e^{r t} (a+d+r)2}{a \left(\Delta t (-r)+e^{\Delta t r}-1\right)+d e^{\Delta t r}-\Delta t r (d+r)-d} .
\end{equation}
Expanding the Taylor series for $\phi^{(k+1)}(t_{n+1})$ around $0$, and 
comparing with the exact solution $\phi(t_{n+1}) = \phi_n e^{(a+d+r)t_{n+1}}$, 
we see that
\begin{multline}
   \phi^{(k+1)}_{n+1} - \phi(t_{n+1}) = \frac{1}{12} \phi_n (a+d) (a+d+r)^2 \Delta t^3 + \frac{1}{12} \phi_n (a+d) (a+d+r)^3 \Delta t^4 \\
      +\frac{1}{720} \phi_n (a+d) (a+d+r)^2 \left(72 r (a+d)+39 (a+d)^2+32 r^2\right) \Delta t^5
      + \mathcal{O}(\Delta t^6).
\end{multline}
In other words, the solution to which the MISDC iterations ``should'' converge 
(given that the residual goes to zero) is locally third-order accurate, 
in the case of a piecewise linear approximation for the advection and diffusion 
terms. I have performed a similar analysis for the case of piecewise cubic, and 
found (as expected) that the error is $\mathcal{O}(\Delta t^5)$.
\end{document}
