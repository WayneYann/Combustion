
Consider 
\begin{eqnarray}
  \label{eq:phiode}
  \phi'(t)& = & (a+d+r)\phi\\
  \phi(t_n)& = & \phi_n,
\end{eqnarray}
or equivalently
\begin{equation}
  \label{eq:phipicard}
  \phi(t) =  \phi_n+ \int_{t_n}^t (a+d+r)\phi(\tau) d\tau.
\end{equation}

Let's assume that we have an approximate solution $\phik(t)$
with $\phik(t_n) = \phi_n$.

We define the residual
\begin{equation}
  \label{eq:resid}
  \Res(\phik,t) = \phi_n + \int_{t_n}^t (a+d+r)\phik(\tau) d\tau -\phik(t) 
\end{equation}
We define the correction,
\begin{eqnarray}
  \c(t)& = & \int_{t_n}^t a(\phik(t)+\c(t))- a(\phik(t)) d\tau \\
        & + &   \int_{t_n}^t d( \phik(t)+\c(t)) - d(\phik(t))d\tau  \\
        & +  & \int_{t_n}^t r(\phik(t)+\c(t)) - r(\phik(t))d\tau \\
       &  +  &  \Res(\phik,t) 
\end{eqnarray}
For the linear problem this simplifies to:
\begin{equation}
  \c(t) =  \int_{t_n}^t a\c(t) d\tau + \int_{t_n}^t d \c(t) d\tau   +  \int_{t_n}^t r\c(t) d\tau   +  \Res(\phik,t) \end{equation}

We will do a coupled splitting of this equation by defining
\begin{eqnarray}
  \cA(t) & = & \int_{t_n}^t a\cA(t) d\tau + \Res(\phik,t)
\end{eqnarray}

\begin{eqnarray}
  \cAD(t) & = &  \int_{t_n}^t a\cA(t) + d\cAD(t) d\tau + \Res(\phik,t) \\
          & = &  \cA(t) +  \int_{t_n}^t d\cAD(t) d\tau 
\end{eqnarray}
and
\begin{eqnarray}
  \ck(t)& = & \int_{t_n}^t a\cA(t) + d\cAD(t) +  r\c^k(t) d\tau   +   \Res(\phik,t)  \\
         & = &  \cAD(t) +  \int_{t_n}^t r\c^k(t) d\tau \\
\end{eqnarray}

To discretize these equations, we must first discretize $\Res(\phik,t)$ and then
decide how to discretize each equation in turn.
First we approximate the residual by applying the trapezoid rule to the $a$ and $d$ terms
and leaving the $r$ integral ``exact''.  This is equivalent to assuming a linear profile
of the $a$ and $d$ terms
\begin{equation}
  \label{eq:residtil}
  \Restil(\phik,t)=  \phi_n + \int_{t_n}^t \ell_k(\tau) +  \int_{t_n}^t r\phik(\tau) d\tau -\phik(t) 
\end{equation}
where 
\begin{equation} \label{eq:ell}
  \ell_k(t) = (a+d)(\phikn+ \frac{t-t_n}{\Dt} \phiknp) = (a+d)(\phi_n\frac{t_{n+1}-t}{\Dt} + \frac{t-t_n}{\Dt} \phiknp)
\end{equation}
So doing the integral gives
\begin{equation}
  \label{eq:residtilII}
  \Restil(\phik,t)=  \phi_n + (a+d)(\phikn(t-t_n)+ \frac{(t-t_n)^2}{2\Dt} \phiknp) +  \int_{t_n}^t r\phik(\tau) d\tau -\phik(t) 
\end{equation}
In particular 
\begin{equation}
  \label{eq:residtilII}
  \Restil(\phik,t_{n+1})=  \phi_n + \frac{(a+d)\Dt}{2}(\phi_n +  \phiknp)+  \int_{t_n}^t r\phik(\tau) d\tau -\phik(t) 
\end{equation}
We leave for a moment the discussion of the exact integral piece. 
If we treat the integral for $\cA$ with forward Euler 
\begin{equation}
  \cA(t_{n+1})  = \cA(t_{n})+\Dt a\cA(t_n)  + \RestilAD(\phik,t_{n+1}) =  \RestilAD(\phik,t_{n+1}) 
\end{equation}
where now we have introduced $\RestilAD(\phik,t_{n+1})$ as the manner in which  
$\Restil(\phik,t_{n+1})$ is approximated.
Likewise, approximating $\cAD(t_{n+1})$ with backward Euler gives
\begin{equation} \label{eq:cAD}
  \cAD(t_{n+1})  = \cAD(t_{n}) + d\Dt\cAD(t_{n+1})+ \RestilAD(\phik,t_{n+1}) = \frac{\RestilAD(\phik,t_{n+1})}{1-d \Delta t} 
\end{equation}
Now we want to turn the equation for $\ck(t)$ into an ODE by differentiating something.
One choice is to go back to the original definition of $\ck(t)$ and write
\begin{eqnarray}
  (\ck)'(t) = a\cA(t) + d\cAD(t) +  r\c^k(t) + \Restil'(\phik,t)
\end{eqnarray}
differentiating the definition (\ref{eq:residtil}) gives
\begin{eqnarray}
  (\ck)'(t) = a\cA(t) + d\cAD(t) + r\c^k(t) +  \ell_k(t) +   r\phik(t)  -(\phik)'(t).
\end{eqnarray}
Rearranging slightly gives  
\begin{eqnarray}
  (\ck+\phik)'(t) = a\cA(t) + d\cAD(t)  + r(\c^k+\phik)(t) +  \ell_k(t). 
\end{eqnarray}
This is now an ODE for the new solution $\ck+\phik$.

We are still left with the question of how to define 
$a\cA(t) + d\cAD(t)$.  By setting the first to a ``forward Euler''
approximation $a\cA(t_n) = 0$, we are left with just the 
$d\cAD(t)$  We define implicitly  
\begin{equation}
  \label{eq:phiAD}
  \phiAD = \phik(t)+\cAD(t)
\end{equation}
and apply a ``backward Euler'' rule so that
\begin{equation}
  \label{eq:phiAD}
  \cAD(t)= \phiADnp - \phiknp
\end{equation}
Now we can get to Will's Eq (7) by approximating $\ell_k(t)$
by a peicewise contstant term
\begin{eqnarray}
  (\c^k+\phik)'(t) =   r(\c^k+\phik)(t)+  d\phiADnp - d\phiknp
+  \frac{1}{2} (\ell_k(t_n)+\ell_k(t_{n+1})).  
\end{eqnarray}

This required a dubious mix of cancellation that is more apparent
if I stick with
\begin{eqnarray}
  (\ck+\phik)'(t) = a\cA(t) + d\cAD(t)  + r(\c^k+\phik)(t) +  \ell_k(t). 
\end{eqnarray}
There are several definitions of $\cAD(t)$ that could be derived
from \ref{eq:cAD}, for example 
\begin{eqnarray}
  d\cAD(t) = (\frac{t-t_n}{\Dt})d \cAD(t_{n+1})
\end{eqnarray}
would be a linear representation.  We could get more fancy and write
\begin{eqnarray}
  d\cAD(t) = \frac{\RestilAD(\phik,t_{n+1})}{1-d t}
\end{eqnarray}

Some ``conclusions''
\begin{enumerate}
\item If you look to consistently represent terms in both the residual
and the correction equations as linear functions, you get something different
than might if you cancel terms in different places using different representations
of the integrals
\item  I still haven't said how the $\RestilAD(\phik,t_{n+1})$ is going to be computed
(or $\RestilAD(\phik,t_{n+1})$)
\item  The formalism here be easier to make general by increasing the order of the 
polynomial $\ell_k$ to something higher order
\item  In this formulation, if we get $\Restil(\phik,t_{n+1}) = 0$ then more iterations
will just keep it there.
\item  It is not clear to me what function would satisfy $\Restil(\phik,t_{n+1}) = 0$
given a exact treatment of the $r$ integral and trapezoid rule of the other two.
I have tried to find a closed form, but can't.
\end{enumerate}





 




%%% Local Variables:
%%% mode: latex
%%% TeX-master: "paper"
%%% End:
